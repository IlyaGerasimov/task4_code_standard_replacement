\documentclass{article}
\usepackage{graphicx}
\usepackage[russian]{babel}
\usepackage[T1,T2A]{fontenc}
\usepackage[utf8]{inputenc}
\usepackage{amsmath, amsfonts}

\title{Задание 4}
\author{И. Герасимов}
\date{}
\begin{document}
\maketitle

\section{Использование программы}

\begin{verbatim}
python3.8 main.py [-h] [-g -r [R] -n [N] -p [PROBABILITY]]
[-c -f [FILE] -m [MESSAGE] [-e [ERROR]]] [-d -f [FILE] -y [Y]] 
\end{verbatim}

\subsection{Генерация кода (указан флаг \texttt{-g})}\label{gen}

\begin{itemize}
\item \texttt{-r} --- число проверочных символов;
\item \texttt{-n} --- желаемая максимальная длина блока сообщения, передаваемая по каналу связи;
\item \texttt{-p} --- вероятность ошибки в канале связи для двоичного симметричного канала.
\end{itemize}

Если указанные параметры не позволяют сформировать код, удовлетворяющий прямой теореме Шеннона, то будет выполняться понижение $n$, пока не требования не будут соблюдены или не будет исчерпано множество возможных $n$.
Во атором случае работа программы закончиться с выводом того, что невозможно найти подходящие параметры относительно прямой теоремы Шеннона.

Будут созданы 3 файла (в конце каждого файла указывается индекс, чтобы избежать перезаписываний):

\begin{itemize}
\item \texttt{info} --- полная информация о построенном коде;
\item \texttt{code} --- информация для кодера;
\item \texttt{decode} --- информация для декодера.
\end{itemize}

\subsection{Режим кодирования (указан \texttt{-c})}

\begin{itemize}
\item \texttt{-f} - файл для кодера (например, второй файл \texttt{code}, получаемый при генерации кода);
\item \texttt{-m} - кодируемое сообщение;
\item \texttt{-e} - ошибка передачи по каналу связи.
Если не указана, то генерируется случайная.
\end{itemize}

Будет выполнен вывод сначала результата кодирования, затем самой ошибки.

\subsection{Режим декодирования (указан \texttt{-d})}

\begin{itemize}
\item \texttt{-f} - файл для декодера (например, третий файл \texttt{decode}, получаемый при генерации кода);
\item \texttt{-y} - сообщение из канала связи;
\end{itemize}

Если получено сообщение с ошибкой меньше веса $t$ определяемого кодом, то будет выведен результат и утверждение с ошибкой и её весом о том, что ошибки декодирования нет.

Иначе выводится сообщение о декодировании через стандартное расположение (лидера смежного класса, полученного по синдрому) и вероятность ошибки декодирования.

\section{Описание работы генерации кода}

\begin{enumerate}
\item Определяется максимальная скорость кодирования и соответствующие $k, n$ через прямую теорему Шеннона;
\item Определяется максимально возможное минимальное расстояние кода $d$ через неравенство Варшамова-Гильберта;
\item Определяется количество исправляемых ошибок $t$ через $d$;
\item Вычисляются проверочная и порождающая матрицы $H, G$ в систематическом виде ($sigma$ --- тождественная подстановка);
\item Строится стандартное расположение;
\item Строится таблица синдромов для векторов веса не больше $t$, а также таблица синдромов для лидеров смежных классов по стандартному расположению;
\item Вычисляется ошибка декодирования по стандартному расположению.
Из-за выбора параметров кода вероятность ошибки декодирования для получаемых из канала сообщений с вектором ошибки веса не больше t равна 0;
\item Формируются файлы, описанные в \ref{gen}.
\end{enumerate}

\textbf{Замечание:}
Поскольку в задании рассматривается информация относительно векторов ошибки веса не больше $t$ и стандартное расположение, вообще говоря, можно было бы не строить.
Достаточно взять все вектора, веса не больше $t$, посчитать их синдромы и сохранять только эту таблицу.
Однако, в режиме кодирования $e$ может быть любым, поэтому необходимо использовать стандартное расположение, чтобы снизить ошибку декодирования.

\section{Описание работы кодера}

Выполняется умножение $x = mG$ и накладывается ошибка $y = x \oplus e$.

\subsection{формат файла кодера}

\begin{enumerate}
\item значение $r$;
\item значение $n$;
\item значение ошибки декодирования;
\item порождающая матрица $G$;
\end{enumerate}

\section{Описание работы декодера}

\begin{enumerate}
\item Вычисляется синдром полученного сообщения: $S(y) = S(e) = H^Te$.
\item Выполняется поиск синдрома в таблице синдромов для векторов ошибки веса не больше $t$;
\item Для найденого синдрома выполняется проверка, что запись не пуста и содержит только один вектор ошибки.
Если это так, то вектором ошибки берется эта запись;
\item Если же условия не выполнены, то реализуется декодирование по стандартному расположению и берется лидер смежного класса, соответствующий этому синдрому;
\item Вычисляется вектор $x = y \oplus e$ и берется его начальная часть по причине того, что используется матрица в систематическом виде.
\item Если в качестве вектора ошибки была взята запись из таблицы синдромов для векторов веса не больше $t$, то выводится результат и информация о том, что вероятность ошибки равна 0.
Иначе выводится результат и вероятность ошибки декодирования.
\end{enumerate}

\end{document}